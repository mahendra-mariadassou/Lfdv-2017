\documentclass[a4paper,12pt]{article}
%%%%%%%%%%%%%%%%%%%%%%%%%%%%%%%%%%%%%%%%%%%%%%%%%%%%%%%%%%%%%%%%%%%%
%%% ZONE ROUGE : intervention très fortement deconseillee
\usepackage[utf8]{inputenc} % accent dans la source
\usepackage[T1]{fontenc}
\usepackage{times}
\usepackage[french]{babel}
\usepackage{amsmath}
\usepackage{amsfonts}
\usepackage{amssymb}
\usepackage{fancybox}
\usepackage{pstricks}
\usepackage{pst-plot}
\usepackage{pst-char}
\usepackage{pst-text}
% \usepackage{psfig}
\usepackage{ifthen}
\def\R{\mathbb{R}}
\def\C{\mathbb{C}}
\def\Z{\mathbb{Z}}
\def\Q{\mathbb{Q}}
\def\N{\mathbb{N}}
\def\F{\mathbb{F}}
\def\P{\mathbb{P}}
\def\A{\mathbb{A}}

%%%% Théorème, Proposition et tout le reste%%%%
\newcommand{\proofbegin}{\paragraph{Proof.}}
\newcommand{\proofend}{$\blacksquare$\bigskip}
\newtheorem{theorem}{Théorème}
\newtheorem{proposition}[theorem]{Proposition}
\newtheorem{definition}[theorem]{Définition}
\newtheorem{lemma}[theorem]{Lemme}
\newtheorem{corollary}[theorem]{Corollaire}


%%%%%%%%%%%%%%%%%%%%%%%%%%%%%%%%%%%%%%%%%%%%%%%%%%%%%%%%%%%%%%%%%%%%%
%%% ZONE ORANGE : intervention deconseillee
\parindent=0pt
\textwidth 17.0cm
\textheight25.0cm
\hoffset-1.0cm
\voffset-3.0cm
%%%%%%%%%%%%%%%%%%%%%%%%%%%%%%%%%%%%%%%%%%%%%%%%%%%%%%%%%%%%%%%%%%%%%
%%% ZONE VERTE : Intervention obligatoire
%%% ADAPTER SUIVANT LA NATURE DE L'EPREUVE
\def\Exam{Feuille d'Exercices \og Rappels de cours \fg}
\def\Date{\today}
\def\Classe{TS3}

%%%%%%%%%%%%%%%%%%%%%%%%%%%%%%%%%%%%%%%%%%%%%%%%%%%%%%%%%%%%%%%%%%%%%
%%% ZONE VERTE : 
%%% ALIGNER LES EQUATIONS A GAUCHE
\makeatletter
\newenvironment*{fleqn}{
    \@fleqntrue
    \setlength\@mathmargin{0pt}%
    \ignorespaces
}{%
    \ignorespacesafterend
}
\makeatother



%%%%%%%%%%%%%%%%%%%%%%%%%%%%%%%%%%%%%%%%%%%%%%%%%%%%%%%%%%%%%%%%%%%%%
\begin{document}
\newcounter{nexo}
\setcounter{nexo}{1}
\newcommand{\Exo}{\medskip
  {\bf Exercice \arabic{nexo} : }
  \addtocounter{nexo}{1}}
\newcommand{\Pb}{{\bf Problème \arabic{nexo} : } 
\addtocounter{nexo}{1} \bigskip}
%%%%%%%%%%%%%%%%%%%%%%%%%%%%%%%%%%%%%%%%%%%%%%%%%%%%%%%%%%%%%%%%%%%%%
%%% ZONE BLEUE : Intervention parfois utile mais a faire prudemment
{\bf  \hfill \Date \quad ~}
%
\vskip 1cm
%
%%% FAIRE UN CHOIX (3 choix possibles)
%%% 1)
%\centerline{\psframebox[fillstyle=solid,fillcolor=lightgray]
%\textbf{\LARGE \black \Exam}}
%%% 2)
%\centerline{\psframebox{\bf \LARGE  \Exam}}
%%% 3)
\centerline{\bf \LARGE \Exam}
%
\vskip 1.5cm
%


%%%%%%%%%%%%%%%%%%%%%%
%%% CORPS DU SUJET %%%
%%%%%%%%%%%%%%%%%%%%%%

%% \section*{Feuille d'exercices 1}

\Exo \textbf{Résolution d'équations}

Résoudre les équations suivantes:
\begin{enumerate}
\item $x^2 + 3x + 40 = 0$
\item $6x^4 - 5x^3 - 4x^2 = 0$
\item $4x^6 + 10x^5 + x^4 = 0$
\item $x^7 + 6x^4 - 16x = 0$ (Astuce: racine évidente + changement de variables)
\item $x^{1/2} - 8x^{1/4} -15 = 0$
\item $\frac{x}{4x + 5} + \frac{3x}{x - 8} = 0$
\end{enumerate}

% \Exo \textbf{Domaine de définition, domaine image}

% Déterminer le domaine $\mathcal{D}_f$ et le domaine image (c'est-à-dire $f(\mathcal{D}_f)$) des fonctions suivantes:

% \begin{enumerate}
% \item $f(x) = x^2 - 8x + 3$
% \item $f(x) = 5 - \sqrt{2x}$
% \item $f(x) = 1 + \sqrt{6 - 7x}$
% \item $f(x) = 12 + 9|x^2 - 1|$ (Astuce: racine évidente + changement de variables)
% \item $f(x) = x^{1/2} - 8x^{1/4} -15$
% \item $f(x) = \frac{x}{4x + 5} + \frac{3x}{x - 8}$
% \end{enumerate}

% \Exo \textbf{Domaine de définition}

% Déterminer le domaine $\mathcal{D}_f$ des fonctions suivantes:

% \begin{enumerate}
% \item $f(x) = \frac{8x^2 - 12x + 4}{16x +9}$
% \item $f(x) = \frac{3x+1}{5x^2 -3x -2}$
% \item $f(x) = \frac{x^2 + x}{x^3 - 9x^2 + 2x}$
% \item $f(x) = \sqrt{4x^3 -4x + x}$
% \item $f(x) = \frac{x+1}{\sqrt{x^4 - 6x^3 + 9x^2}}$
% \item $f(x) = \frac{8}{x^2 -3x - 4} + \frac{3}{\sqrt{12 - 7x - 3x^2}}$
% \end{enumerate}

% \Exo \textbf{Composition}

% Pour chaque paire de fonctions $(f,g)$, calculer $g\circ f$ et $f \circ g$:

% \begin{enumerate}
% \item $f(x) = 2x + 5$, $g(x) = -23x + 8$
% \item $f(x) = 2x^2 + x - 4$, $g(x) = -x^2 + 7x$
% \item $f(x) = \frac{x}{2x+3}$, $g(x) = 5x + 8$
% \end{enumerate}

% \Exo \textbf{Réciproque}

% Calculer les fonctions réciproques des fonctions suivantes:
% \begin{enumerate}
% \item $f(x) = 11x - 8$
% \item $f(x) = \sqrt{15x + 2}$
% \item $f(x) = (2x + 1)^3 + 7$
% \item $f(x) = \frac{x - 1}{-12x + 9}$
% \end{enumerate}


% \Exo \textbf{Logarithme}

% Déterminer (sans calculatrice) les fonctions suivantes:
% \begin{enumerate}
% \item $\log_{7}(343)$
% \item $\log_{4}(1024)$
% \item $\log_{11}(1/121)$
% \item $\log_{16}(4)$
% \item $\log(1000)$
% \end{enumerate}

% \Exo \textbf{Logarithme}

% Développer les expressions suivantes:
% \begin{enumerate}
% \item $\log(10a^7b^3c^{-8})$
% \item $\log(z^2(x^2 + 4)^3)$
% \item $\log \left( \frac{w^2\sqrt[4]{t^3}}{\sqrt{t + w}} \right)$
% \end{enumerate}

\Exo \textbf{\'Equations exponentielles et logarithmiques}

Résoudre dans $\mathbb{R}$ les équations suivantes:
\begin{enumerate}
\item $12 + 5\exp(10x - 7) = 15$
\item $4\exp(2x + x^2) - 7 = 2$
\item $4x^2 - 3x^2\exp(2 - x) = 0$ (Astuce: factoriser)
\item $16 + 4\ln(x + 2) = 7$
\item $\ln{3x+1}  - \ln{x} = -2$
\item $2\ln(x) - \ln(x^2 +4x + 1) = 0$
\item $11 - 5^{9x -1} = 3$
\item $1 + 3^{x^2 - 2} = 5$
\end{enumerate}

\Exo \textbf{Résolution d'équations}

Résoudre dans $\mathbb{R}$ les équations suivantes:
\begin{enumerate}
\item $e^x + e^{-x} = 2$
\item $(\ln{x})^2 + 3\ln{x} + 2 = 0$
\item $x = \sqrt{x} + 2$
\item $x^2 - 3x + 4 + \frac{8 - 6x}{x^2 - 2} = 0$
\end{enumerate}

\Exo \textbf{Résolution d'inéquations}

Résoudre dans $\mathbb{R}$ les inéquations suivantes:
\begin{enumerate}
\item $\ln(3x) < \ln(2x)$
\item $3 \times 2^{3x -4} \geq 7^8$
\item $5 \left( \frac{1}{3} \right)^x \leq 10^{-10}$
\item $\sqrt{x} \geq x + 1$
\end{enumerate}

\Exo \textbf{Injections, surjections, bijections}

Parmi les fonctions suivantes, lesquelles sont des injections/surjections/bijections?
\begin{enumerate}
\item $f: \mathbb{R} \to \mathbb{R}$ définie par $f(x) = e^x$
\item $f: \mathbb{R} \to \mathbb{R}_+^\star$ définie par $f(x) = e^x$
\item $f: \mathbb{R} \to \mathbb{R}_+$ définie par $f(x) = x^2$
\item $f: \mathbb{R}_+ \to \mathbb{R}_+$ définie par $f(x) = x^2$
\item $f: \mathbb{R} \to \mathbb{R}$ définie par $f(x) = \begin{cases} x-1 \text{ si } x \leq 0 \\ x+1 \text{ si } x > 0 \end{cases}$
\item $f: \mathbb{N} \to \mathbb{N}$ définie par $f(n) = 2n$
\item $f: \mathbb{N} \to \mathbb{Z}$ définie par $f(n) = \begin{cases} \lfloor n/2 \rfloor \text{ si } n \text{ pair} \\ - \lfloor n/2 \rfloor \text{ si } n \text{ impair} \end{cases}$
\end{enumerate}


\Exo \textbf{Récurrence}

Montrer les formules closes suivantes par récurrence:
\begin{enumerate}
\item $S_n = 1 + 2 + \dots + n = \frac{n(n+1)}{2}$
\item $S_n = 1^2 + 2^2 + \dots + n^2 = \frac{n(n+1)(2n+1)}{6}$
\item $S_n = 1 + q + q^2 + \dots + q^n = \frac{1-q^{n+1}}{1-q}$ pour tout $q \in \mathbb{R}-\{1\}$
\end{enumerate}

\Exo \textbf{\'Equations trigonométriques}

Résoudre dans $\mathbb{R}$ (sauf mention explicite du contraire) les équations trigonométriques suivantes:
\begin{enumerate}
\item $10\cos(8\theta) = -5$
\item $2\sin(\theta/4) = \sqrt{3}$
\item $2\sin(\theta/4) = \sqrt{3}$ dans $[0, 16\pi]$
\item $10 + 7\tan(4\theta) = 3$ dans $[-\pi, 0]$. 
\item $3 - 4\sin(4\theta) = 5$ dans $[-3\pi/2, -\pi/2]$
\end{enumerate}

\Exo \textbf{Quantificateurs}

Considérons les propositions suivantes:
\begin{align*}
(P) &  \forall \varepsilon > 0, \exists \delta > 0 \quad \text{tel que} \quad \forall x \in (a-\delta, a+\delta), |f(x) - f(a)| < \varepsilon \\
(Q) &  \exists \delta > 0 \quad \text{tel que} \quad \forall \varepsilon > 0, \forall x \in (a-\delta, a+\delta), |f(x) - f(a)| < \varepsilon \\
\end{align*}

\begin{enumerate}
\item De \og qui\fg{} parlent ces propositions? En particulier, quels paramètres nécessitent un contexte?
\item L'une des propositions implique-t-elle l'autre? Sont-elles équivalentes?
\item Donner la négation des deux propositions.
\item Que signifient ces propositions (en langage naturel)?
\end{enumerate}

\end{document}