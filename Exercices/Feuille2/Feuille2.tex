\documentclass[a4paper,12pt]{article}
%%%%%%%%%%%%%%%%%%%%%%%%%%%%%%%%%%%%%%%%%%%%%%%%%%%%%%%%%%%%%%%%%%%%
%%% ZONE ROUGE : intervention très fortement deconseillee
\usepackage[utf8]{inputenc} % accent dans la source
\usepackage[T1]{fontenc}
\usepackage{times}
\usepackage[french]{babel}
\usepackage{amsmath}
\usepackage{amsfonts}
\usepackage{amssymb}
\usepackage{fancybox}
\usepackage{pstricks}
\usepackage{pst-plot}
\usepackage{pst-text}
% \usepackage{psfig}
\usepackage{ifthen}
\def\R{\mathbb{R}}
\def\C{\mathbb{C}}
\def\Z{\mathbb{Z}}
\def\Q{\mathbb{Q}}
\def\N{\mathbb{N}}
\def\F{\mathbb{F}}
\def\P{\mathbb{P}}
\def\A{\mathbb{A}}

%%%% Théorème, Proposition et tout le reste%%%%
\newcommand{\proofbegin}{\paragraph{Proof.}}
\newcommand{\proofend}{$\blacksquare$\bigskip}
\newtheorem{theorem}{Théorème}
\newtheorem{proposition}[theorem]{Proposition}
\newtheorem{definition}[theorem]{Définition}
\newtheorem{lemma}[theorem]{Lemme}
\newtheorem{corollary}[theorem]{Corollaire}


%%%%%%%%%%%%%%%%%%%%%%%%%%%%%%%%%%%%%%%%%%%%%%%%%%%%%%%%%%%%%%%%%%%%%
%%% ZONE ORANGE : intervention deconseillee
\parindent=0pt
\textwidth 17.0cm
\textheight25.0cm
\hoffset-1.0cm
\voffset-3.0cm
%%%%%%%%%%%%%%%%%%%%%%%%%%%%%%%%%%%%%%%%%%%%%%%%%%%%%%%%%%%%%%%%%%%%%
%%% ZONE VERTE : Intervention obligatoire
%%% ADAPTER SUIVANT LA NATURE DE L'EPREUVE
\def\Exam{Exercices \og Formules trigonométriques\fg{}}
\def\Date{\today}
\def\Classe{TS3}

%%%%%%%%%%%%%%%%%%%%%%%%%%%%%%%%%%%%%%%%%%%%%%%%%%%%%%%%%%%%%%%%%%%%%
%%% ZONE VERTE :
%%% ALIGNER LES EQUATIONS A GAUCHE
\makeatletter
\newenvironment*{fleqn}{
    \@fleqntrue
    \setlength\@mathmargin{0pt}%
    \ignorespaces
}{%
    \ignorespacesafterend
}
\makeatother



%%%%%%%%%%%%%%%%%%%%%%%%%%%%%%%%%%%%%%%%%%%%%%%%%%%%%%%%%%%%%%%%%%%%%
\begin{document}
\newcounter{nexo}
\setcounter{nexo}{1}
\newcommand{\Exo}{\medskip
  {\bf Exercice \arabic{nexo} : }
  \addtocounter{nexo}{1}}
\newcommand{\Pb}{{\bf Problème \arabic{nexo} : }
\addtocounter{nexo}{1} \bigskip}
%%%%%%%%%%%%%%%%%%%%%%%%%%%%%%%%%%%%%%%%%%%%%%%%%%%%%%%%%%%%%%%%%%%%%
%%% ZONE BLEUE : Intervention parfois utile mais a faire prudemment
{\bf  \hfill \Date \quad ~}
%
\vskip 1cm
%
%%% FAIRE UN CHOIX (3 choix possibles)
%%% 1)
%\centerline{\psframebox[fillstyle=solid,fillcolor=lightgray]
%\textbf{\LARGE \black \Exam}}
%%% 2)
%\centerline{\psframebox{\bf \LARGE  \Exam}}
%%% 3)
\centerline{\bf \LARGE \Exam}
%
\vskip 1.5cm
%


%%%%%%%%%%%%%%%%%%%%%%
%%% CORPS DU SUJET %%%
%%%%%%%%%%%%%%%%%%%%%%

% \section*{Corrigé du devoir 1}

\Exo \textbf{Formules trigonométriques (I)}

Calculer les valeurs exactes des quantités suivantes:
\begin{enumerate}
\item $\cos(\pi/12)$
\item $\sin(11\pi/12)$
\item $\cos(\pi/8)$
\item $\sin(7\pi/8)$
\end{enumerate}

\Exo \textbf{Formules trigonométriques (II)}

Résoudre dans $\R$ les (in)équations suivantes

\begin{enumerate}
\item $\cos(x) + \sin(x) \geq 1$
\item $\cos(x) + \sqrt{3}\sin(x) \geq 1$
\item $\cos(2x) + 2\sin(x) =  0$
\item $\sin(2x) - 2\sin(x) =  0$
\item $\cos(x) + \cos(2x) + \cos(3x) = 0$
\item $\cos(3x) - \sin(2x) = 0$ [difficile]
\end{enumerate}

\Exo \textbf{Tangente}

Donner le nombre de solutions dans $[0, \pi]$ de l'équation
\[
\tan(x) + \tan(2x) + \tan(3x) + \tan(4x) = 0
\]

\Exo \textbf{Fonctions trigonométriques réciproques}

Résoudre dans $\mathbb{R}$ (sauf mention explicite du contraire) les équations trigonométriques suivantes:
\begin{enumerate}
\item $10\cos(8\theta) = -5$
\item $2\sin(\theta/4) = \sqrt{3}$
\item $2\sin(\theta/4) = \sqrt{3}$ dans $[0, 16\pi]$
\item $10 + 7\tan(4\theta) = 3$ dans $[-\pi, 0]$.
\item $3 - 4\sin(4\theta) = 5$ dans $[-3\pi/2, -\pi/2]$
\item $2\cos^2(x) - 3\cos(x) + 1 = 0$ dans $[0, 2\pi]$
\end{enumerate}

\Exo \textbf{Inéquations}

Résoudre dans $\mathbb{R}$ (sauf mention explicite du contraire) les équations suivantes:
\begin{enumerate}
\item $|\cos(x)| \geq |\sin(x)|$
\item $\ln(\cos^2(x)) = 0$
\item $2\ln(\cos(x)) = 0$
\item $\sqrt{1 - \cos^2(x)} = \frac{\sqrt{3}}{2}$
\item $e^{\cos(x)} \leq 1$
\end{enumerate}

\Exo {$\boldsymbol{\arcsin}$}

On cherche à calculer $X = \displaystyle \arcsin\left( - \sqrt{\frac{2 - \sqrt{2}}{4}}\right)$.
\begin{enumerate}
\item Montrer que pour tout $x \in \mathbb{R}$
  \begin{equation*}
    \sin^2(x)  = \frac{1 - \cos(2x)}{2}
  \end{equation*}
\item Appliquer la formule précédente à $x = \frac{\pi}{8}$.
\item En déduire la valeur de $X$.
\item Vérifier que vous n'avez pas fait de fautes, par exemple avec une calculatrice.
\end{enumerate}

\Exo \textbf{Produit de cosinus}

Soit $a \in (0, \pi)$. Calculer pour tout $n \in \N^\star$
\[
\prod_{k=1}^n \cos\left(\frac{a}{2^k}\right)
\]
On pourra utiliser $\sin(2x) = 2\cos(x)\sin(x)$. En déduire
\[
\lim_{n \to \infty} \sum_{k=1}^n \ln\left( \cos \left( \frac{a}{2^k} \right) \right)
\]

\textbf{Correction de $\cos(2x) + 2\sin(x) = 0$}

\begin{align*}
\cos(2x) + 2\sin(x) = 0 &
\Leftrightarrow (\cos^2(x) - \sin^2(x)) +2\sin(x) = 0 \\
\Leftrightarrow (1 - 2\sin^2(x)) +2\sin(x) = 0
\Leftrightarrow -2\sin^2(x)) +2\sin(x) + 1 = 0
\end{align*}

On pose $\sin(x) = X$ et on cherche à résoudre l'équation de degré 2 $-2X^2 + 2X + 1 = 0$. En calculant le discriminant, on trouve que les solutions sont $X_1 = \frac{1-\sqrt{3}}{2}$ et $X_2 = \frac{1+\sqrt{3}}{2}$ et on est ramené à résoudre $\sin(x) = X_1$ et $\sin(x) = X_2$. $X_2 > 1$ donc l'équation $\sin(x) = X_2$ n'a pas de solution. $X_1 \in [-1, 1]$ donc l'équation $\sin(x) = X_1$ a une infinité de solutions. Comme $X_1$ n'est pas un $\sin$ remarquable, on passe par la fonction $\arcsin$

\begin{align*}
\sin(x) = X_1 & \Leftrightarrow \sin(x) = \sin(\arcsin(X_1)) \Leftrightarrow \begin{cases} x = \arcsin(X_1) + 2k\pi \quad (k \in \mathbb{Z}) \\ x = \pi - \arcsin(X_1) + 2k\pi \quad (k \in \mathbb{Z}) \end{cases} \\
& \Leftrightarrow \begin{cases} x = \arcsin\left( \frac{1 - \sqrt{3}}{2} \right) + 2k\pi \quad (k \in \mathbb{Z}) \\ x = \pi - \arcsin\left( \frac{1 - \sqrt{3}}{2} \right) + 2k\pi \quad (k \in \mathbb{Z}) \end{cases}
\end{align*}

\textbf{Correction de $\cos(3x) - \sin(2x) = 0$}

On cherche à tout exprimer en fonction de $\cos(x)$ et $\sin(x)$. On utilise le fait que $\sin(2x) = 2\sin(x)\cos(x)$ et $\cos(2x) = \cos^2(x) - \sin^2(x)$
\begin{align*}
\cos(3x) & = \cos(2x + x) = \cos(2x)\cos(x) - \sin(x)\sin(2x) \\
         & = (\cos^2(x) - \sin^2(x))\cos(x) - \sin(x)(2\sin(x)\cos(x)) \\
         & = \cos(x)[ \cos^2(x) - \sin^2(x) - 2\sin^2(x)] = \cos(x)[ 1 - 4\sin^2(x)]
\end{align*}

Donc
\[
\cos(3x) - \sin(2x) = \cos(x)[ 1 - 4\sin^2(x)] - 2\sin(x)\cos(x) = \cos(x)[ 1 - 2\sin(x) - 4\sin^2(x) ]
\]
qui s'annule si $\cos(x) = 0$ (c'est à dire $x = \pi/2 + k\pi$) ou si $1 - 2\sin(x) - 4\sin^2(x) = 0$. On est donc amené à résoudre l'équation de degré 2 $1 - 2X - 4X^2 = 0$ dont les solutions sont $X_1 = \frac{-1 + \sqrt{5}}{4}$ et $X_2 = \frac{-1 - \sqrt{5}}{4}$ puis les équations $\sin(x) = X_1$ et $\sin(x) = X_2$. Comme $X_1$ et $X_2$ sont dans $[-1, 1]$, ces deux équations ont des solutions. Au final, en raisonnant comme précédemment on trouve les solutions suivantes:
\small{
\[
x \in \left\{ -\frac{\pi}{2}, \frac{\pi}{2}, -\arccos\left( \frac{-1 - \sqrt{5}}{4} \right), \arccos\left( \frac{-1 - \sqrt{5}}{4} \right), -\arccos\left( \frac{-1 + \sqrt{5}}{4} \right), \arccos\left( \frac{-1 + \sqrt{5}}{4} \right) \right\} + 2k\pi
\]
}

\textbf{Correction de $10 + 7\tan(4\theta) = 3$ dans $[-\pi, 0]$}

On raisonne par équivalence
\begin{align*}
10 + 7\tan(4\theta) = 3 & \Leftrightarrow \tan(4\theta) = -1 \\
                        & \Leftrightarrow \tan(4\theta) = \tan(-\pi/4) \\
                        & \Leftrightarrow 4\theta = -\pi/4 + k\pi \\
                        & \Leftrightarrow \theta = -\frac{\pi}{16} + k\frac{\pi}{4} \\
\end{align*}

On cherche ensuite les solutions qui sont dans $[-\pi, 0]$. Par exemple, $-\frac{\pi}{16}$ est solution. Comme toutes les solutions sont "décalées" de $\pi/4$, il suffit de lui ajouter $\pi/4$ jusqu'à être plus grand que $0$ et de lui retrancher $\pi/4$ jusqu'à être plus petit que $-\pi$. Au final, on trouve que les solutions sont
\[
x \in \left\{ -\frac{\pi}{16}, -\frac{5\pi}{16}, -\frac{9\pi}{16}, -\frac{13\pi}{16} \right\}
\]

\textbf{Correction de $2\cos^2(x) - 3\cos(x) + 1 = 0$ dans $[0, 2\pi]$}

On pose $X = \cos(x)$ et on se ramène à l'équation de degré $2$ $2X^2 -3X + 1 = 0$ dont les solutions sont $X_1 = -1$ et $X_2 = -1/2$. Les deux valeurs sont dans $[-1, 1]$ dont leur $\arccos$ sont bien définis.

\begin{align*}
\begin{cases}
\cos(x) & = -1 \\
\text{ OU } \\
\cos(x) & = -1/2
\end{cases}
& \Leftrightarrow
\begin{cases}
x & = \pi + 2k\pi \\
\text{ OU }\\
x & = 2\pi/3  + 2k\pi \\
\text{ OU } \\
x & = -2\pi/3  + 2k\pi\\
\end{cases}
\end{align*}

En conservant uniquement les solutions dans $[0, 2\pi]$, on obtient
\[
x \in \left\{ \frac{2\pi}{3}, \pi, \frac{4\pi}{3} \right\}
\]

\textbf{Correction de $|\cos(x)| \geq |\sin(x)|$}

Par $2\pi$-périodicité, on peut se limiter à $x \in [0,2\pi]$. De plus, comme $|\cos(x + \pi)| = |-\cos(x)| = |\cos(x)|$ et de même pour $|\sin|$, la fonction $|\cos(x)| - |\sin(x)|$ est en fait $\pi$-périodique et on peut donc se limiter à $x \in [-\pi/2, \pi/2]$.

On cherche ensuite à se débarrasser des valeurs absolues en distinguant 2 cas.

\textbf{Premier cas:} $\mathbf{x \in [0, \pi/2]}$

Sur cet intervalle, l'inégalité se réduit à $(E_1)$
\[
|\cos(x)| \geq |\sin(x)| \Leftrightarrow \cos(x) \geq \sin(x) \Leftrightarrow \sqrt{2}\cos(x + \pi/4) \geq 0
\]
En faisant (par exemple) une étude de signe de $\cos(x + \pi/4)$ sur $[0, \pi/2]$, on trouve que les solutions de $(E_1)$ sont $x \in [0, \pi/4]$.

\textbf{Second cas:} $\mathbf{x \in [-\pi/2, 0]}$

Sur cet intervalle, l'inégalité se réduit à $(E_2)$
\[
|\cos(x)| \geq |\sin(x)| \Leftrightarrow \cos(x) \geq -\sin(x) \Leftrightarrow \sqrt{2}\cos(x - \pi/4) \geq 0
\]
En faisant (par exemple) une étude de signe de $\cos(x - \pi/4)$ sur $[-\pi/2, 0]$, on trouve que les solutions de $(E_2)$ sont $x \in [-\pi/4, 0]$.

\textbf{Synthèse:}
En recollant les morceaux et en utilisant la $\pi$-périodicité, on obtient que l'ensemble des solutions est $[-\pi/4, \pi/4] + k\pi$


\end{document}

