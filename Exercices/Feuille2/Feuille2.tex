\documentclass[a4paper,12pt]{article}
%%%%%%%%%%%%%%%%%%%%%%%%%%%%%%%%%%%%%%%%%%%%%%%%%%%%%%%%%%%%%%%%%%%%
%%% ZONE ROUGE : intervention très fortement deconseillee
\usepackage[utf8]{inputenc} % accent dans la source
\usepackage[T1]{fontenc}
\usepackage{times}
\usepackage[french]{babel}
\usepackage{amsmath}
\usepackage{amsfonts}
\usepackage{amssymb}
\usepackage{fancybox}
\usepackage{pstricks}
\usepackage{pst-plot}
\usepackage{pst-text}
% \usepackage{psfig}
\usepackage{ifthen}
\def\R{\mathbb{R}}
\def\C{\mathbb{C}}
\def\Z{\mathbb{Z}}
\def\Q{\mathbb{Q}}
\def\N{\mathbb{N}}
\def\F{\mathbb{F}}
\def\P{\mathbb{P}}
\def\A{\mathbb{A}}

%%%% Théorème, Proposition et tout le reste%%%%
\newcommand{\proofbegin}{\paragraph{Proof.}}
\newcommand{\proofend}{$\blacksquare$\bigskip}
\newtheorem{theorem}{Théorème}
\newtheorem{proposition}[theorem]{Proposition}
\newtheorem{definition}[theorem]{Définition}
\newtheorem{lemma}[theorem]{Lemme}
\newtheorem{corollary}[theorem]{Corollaire}


%%%%%%%%%%%%%%%%%%%%%%%%%%%%%%%%%%%%%%%%%%%%%%%%%%%%%%%%%%%%%%%%%%%%%
%%% ZONE ORANGE : intervention deconseillee
\parindent=0pt
\textwidth 17.0cm
\textheight25.0cm
\hoffset-1.0cm
\voffset-3.0cm
%%%%%%%%%%%%%%%%%%%%%%%%%%%%%%%%%%%%%%%%%%%%%%%%%%%%%%%%%%%%%%%%%%%%%
%%% ZONE VERTE : Intervention obligatoire
%%% ADAPTER SUIVANT LA NATURE DE L'EPREUVE
\def\Exam{Exercices \og Formules trigonométriques\fg{}}
\def\Date{\today}
\def\Classe{TS3}

%%%%%%%%%%%%%%%%%%%%%%%%%%%%%%%%%%%%%%%%%%%%%%%%%%%%%%%%%%%%%%%%%%%%%
%%% ZONE VERTE : 
%%% ALIGNER LES EQUATIONS A GAUCHE
\makeatletter
\newenvironment*{fleqn}{
    \@fleqntrue
    \setlength\@mathmargin{0pt}%
    \ignorespaces
}{%
    \ignorespacesafterend
}
\makeatother



%%%%%%%%%%%%%%%%%%%%%%%%%%%%%%%%%%%%%%%%%%%%%%%%%%%%%%%%%%%%%%%%%%%%%
\begin{document}
\newcounter{nexo}
\setcounter{nexo}{1}
\newcommand{\Exo}{\medskip
  {\bf Exercice \arabic{nexo} : }
  \addtocounter{nexo}{1}}
\newcommand{\Pb}{{\bf Problème \arabic{nexo} : } 
\addtocounter{nexo}{1} \bigskip}
%%%%%%%%%%%%%%%%%%%%%%%%%%%%%%%%%%%%%%%%%%%%%%%%%%%%%%%%%%%%%%%%%%%%%
%%% ZONE BLEUE : Intervention parfois utile mais a faire prudemment
{\bf  \hfill \Date \quad ~}
%
\vskip 1cm
%
%%% FAIRE UN CHOIX (3 choix possibles)
%%% 1)
%\centerline{\psframebox[fillstyle=solid,fillcolor=lightgray]
%\textbf{\LARGE \black \Exam}}
%%% 2)
%\centerline{\psframebox{\bf \LARGE  \Exam}}
%%% 3)
\centerline{\bf \LARGE \Exam}
%
\vskip 1.5cm
%


%%%%%%%%%%%%%%%%%%%%%%
%%% CORPS DU SUJET %%%
%%%%%%%%%%%%%%%%%%%%%%

% \section*{Corrigé du devoir 1}

\Exo \textbf{Formule de factorisation}

Soit $\theta \in \mathbb{R}$, l'exponentielle complexe est définie de la façon suivante:
\begin{equation*}
e^{i\theta} = \cos(\theta) + i \sin(\theta)
\end{equation*}

On admet (pour l'instant) que l'exponentielle complexe vérifie les propriétés usuelles de l'exponentielle:
\begin{equation*}
\forall a, b \in \mathbb{R}, e^{i(a+b)} = e^{ia} + e^{ib}
\end{equation*}


\begin{enumerate}
\item En utilisant la formule de l'exponentielle complexe, montrer que pour tout $a,b \in \mathbb{R}$: 
  \begin{align*}
    \cos(a+b) & = \cos(a)\cos(b) - \sin(a)\sin(b) \\
    \sin(a+b) & = \sin(a)\cos(b) + \cos(a)\sin(b) \\
  \end{align*}
\item Soit $p, q, x, y$ des réels. Résoudre le système d'inconnues $x$ et $y$:
  \begin{equation*}
    \begin{cases}
      p & = x + y \\
      q & = x - y \\
    \end{cases}
  \end{equation*}
\item À l'aide du système précédent, exprimer $\cos(a)\cos(b)$ en fonction de $\cos(a+b)$ et de $\cos(a-b)$. Faire de même avec $\sin(a)\sin(b)$. On pourra poser $x = \cos(a)\cos(b)$, $y = \sin(a)\sin(b)$ et choisir $p$ et $q$ astucieusement. 
\item Exprimer $\sin(a)\cos(b)$ en fonction de $\sin(a+b)$ et de $\sin(a-b)$. Faire de même avec $\cos(a)\sin(b)$. 
\end{enumerate}


\Exo {$\boldsymbol{\arcsin}$}

On cherche à calculer $X = \displaystyle \arcsin\left( - \sqrt{\frac{2 - \sqrt{2}}{4}}\right)$. 
\begin{enumerate}
\item Montrer que pour tout $x \in \mathbb{R}$
  \begin{equation*}
    \sin^2(x)  = \frac{1 - \cos(2x)}{2}
  \end{equation*}
\item Appliquer la formule précédente à $x = \frac{\pi}{8}$. 
\item En déduire la valeur de $X$. 
\item (Vérifier que vous n'avez pas fait de fautes, par exemple avec une calculatrice). 
\end{enumerate}



\end{document}