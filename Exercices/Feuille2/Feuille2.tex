\documentclass[a4paper,12pt]{article}
%%%%%%%%%%%%%%%%%%%%%%%%%%%%%%%%%%%%%%%%%%%%%%%%%%%%%%%%%%%%%%%%%%%%
%%% ZONE ROUGE : intervention très fortement deconseillee
\usepackage[utf8]{inputenc} % accent dans la source
\usepackage[T1]{fontenc}
\usepackage{times}
\usepackage[french]{babel}
\usepackage{amsmath}
\usepackage{amsfonts}
\usepackage{amssymb}
\usepackage{fancybox}
\usepackage{pstricks}
\usepackage{pst-plot}
\usepackage{pst-text}
% \usepackage{psfig}
\usepackage{ifthen}
\def\R{\mathbb{R}}
\def\C{\mathbb{C}}
\def\Z{\mathbb{Z}}
\def\Q{\mathbb{Q}}
\def\N{\mathbb{N}}
\def\F{\mathbb{F}}
\def\P{\mathbb{P}}
\def\A{\mathbb{A}}

%%%% Théorème, Proposition et tout le reste%%%%
\newcommand{\proofbegin}{\paragraph{Proof.}}
\newcommand{\proofend}{$\blacksquare$\bigskip}
\newtheorem{theorem}{Théorème}
\newtheorem{proposition}[theorem]{Proposition}
\newtheorem{definition}[theorem]{Définition}
\newtheorem{lemma}[theorem]{Lemme}
\newtheorem{corollary}[theorem]{Corollaire}


%%%%%%%%%%%%%%%%%%%%%%%%%%%%%%%%%%%%%%%%%%%%%%%%%%%%%%%%%%%%%%%%%%%%%
%%% ZONE ORANGE : intervention deconseillee
\parindent=0pt
\textwidth 17.0cm
\textheight25.0cm
\hoffset-1.0cm
\voffset-3.0cm
%%%%%%%%%%%%%%%%%%%%%%%%%%%%%%%%%%%%%%%%%%%%%%%%%%%%%%%%%%%%%%%%%%%%%
%%% ZONE VERTE : Intervention obligatoire
%%% ADAPTER SUIVANT LA NATURE DE L'EPREUVE
\def\Exam{Exercices \og Formules trigonométriques\fg{}}
\def\Date{\today}
\def\Classe{TS3}

%%%%%%%%%%%%%%%%%%%%%%%%%%%%%%%%%%%%%%%%%%%%%%%%%%%%%%%%%%%%%%%%%%%%%
%%% ZONE VERTE :
%%% ALIGNER LES EQUATIONS A GAUCHE
\makeatletter
\newenvironment*{fleqn}{
    \@fleqntrue
    \setlength\@mathmargin{0pt}%
    \ignorespaces
}{%
    \ignorespacesafterend
}
\makeatother



%%%%%%%%%%%%%%%%%%%%%%%%%%%%%%%%%%%%%%%%%%%%%%%%%%%%%%%%%%%%%%%%%%%%%
\begin{document}
\newcounter{nexo}
\setcounter{nexo}{1}
\newcommand{\Exo}{\medskip
  {\bf Exercice \arabic{nexo} : }
  \addtocounter{nexo}{1}}
\newcommand{\Pb}{{\bf Problème \arabic{nexo} : }
\addtocounter{nexo}{1} \bigskip}
%%%%%%%%%%%%%%%%%%%%%%%%%%%%%%%%%%%%%%%%%%%%%%%%%%%%%%%%%%%%%%%%%%%%%
%%% ZONE BLEUE : Intervention parfois utile mais a faire prudemment
{\bf  \hfill \Date \quad ~}
%
\vskip 1cm
%
%%% FAIRE UN CHOIX (3 choix possibles)
%%% 1)
%\centerline{\psframebox[fillstyle=solid,fillcolor=lightgray]
%\textbf{\LARGE \black \Exam}}
%%% 2)
%\centerline{\psframebox{\bf \LARGE  \Exam}}
%%% 3)
\centerline{\bf \LARGE \Exam}
%
\vskip 1.5cm
%


%%%%%%%%%%%%%%%%%%%%%%
%%% CORPS DU SUJET %%%
%%%%%%%%%%%%%%%%%%%%%%

% \section*{Corrigé du devoir 1}

\Exo \textbf{Formules trigonométriques (I)}

Calculer les valeurs exactes des quantités suivantes:
\begin{enumerate}
\item $\cos(\pi/12)$
\item $\sin(\pi/12)$
\item $\cos(\pi/8)$
\item $\sin(\pi/9)$
\end{enumerate}

\Exo \textbf{Formules trigonométriques (II)}

Résoudre dans $\R$ les (in)équations suivantes

\begin{enumerate}
\item $\cos(x) + \sin(x) \geq 1$
\item $\cos(x) + 3\sin(x) \geq 1$
\item $\cos(2x) + 2\sin(x) =  0$
\item $\sin(2x) - 2\sin(x) =  0$
\item $\cos(x) + \cos(2x) + \cos(3x) = 0$
\item $\cos(3x) - \sin(2x) = 0$ [difficile]
\end{enumerate}

\Exo \textbf{Tangente}

Donner le nombre de solutions dans $[0, \pi]$ de l'équation
\[
\tan(x) + \tan(2x) + \tan(3x) + \tan(4x) = 0
\]

\Exo \textbf{Fonctions trigonométriques réciproques}

Résoudre dans $\mathbb{R}$ (sauf mention explicite du contraire) les équations trigonométriques suivantes:
\begin{enumerate}
\item $10\cos(8\theta) = -5$
\item $2\sin(\theta/4) = \sqrt{3}$
\item $2\sin(\theta/4) = \sqrt{3}$ dans $[0, 16\pi]$
\item $10 + 7\tan(4\theta) = 3$ dans $[-\pi, 0]$.
\item $3 - 4\sin(4\theta) = 5$ dans $[-3\pi/2, -\pi/2]$
\item $2\cos^2(x) - 3\cos(x) + 1 = 0$ dans $[0, 2\pi]$
\end{enumerate}

\Exo \textbf{Inéquations}

Résoudre dans $\mathbb{R}$ (sauf mention explicite du contraire) les équations suivantes:
\begin{enumerate}
\item $|\cos(x)| \geq |\sin(x)|$
\item $\ln(\cos^2(x)) = 0$
\item $2\ln(\cos(x)) = 0$
\item $\sqrt{1 - \cos^2(x)} = \frac{\sqrt{3}}{2}$
\item $e^{\cos(x)} \leq 1$
\end{enumerate}

\Exo {$\boldsymbol{\arcsin}$}

On cherche à calculer $X = \displaystyle \arcsin\left( - \sqrt{\frac{2 - \sqrt{2}}{4}}\right)$.
\begin{enumerate}
\item Montrer que pour tout $x \in \mathbb{R}$
  \begin{equation*}
    \sin^2(x)  = \frac{1 - \cos(2x)}{2}
  \end{equation*}
\item Appliquer la formule précédente à $x = \frac{\pi}{8}$.
\item En déduire la valeur de $X$.
\item Vérifier que vous n'avez pas fait de fautes, par exemple avec une calculatrice.
\end{enumerate}

\Exo \textbf{Produit de cosinus}

Soit $a \in (0, \pi)$. Calculer pour tout $n \in \N^\star$
\[
\prod_{k=1}^n \cos\left(\frac{a}{2^k}\right)
\]
On pourra utiliser $\sin(2x) = 2\cos(x)\sin(x)$. En déduire
\[
\lim_{n \to \infty} \sum_{k=1}^n \ln\left( \cos \left( \frac{a}{2^k} \right) \right)
\]


\end{document}
